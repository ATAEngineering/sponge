\documentclass{article}
%% Change fonts to something friendly to acrobat
%%
%\usepackage{courier}
%\usepackage{times,mathptmx}
%\DeclareMathAlphabet{\bm}{OT1}{ptm}{b}{it}
\usepackage{graphicx}
\usepackage{amsmath} % used for extended formula formatting tools
\usepackage{amssymb}
\usepackage{theorem}
\usepackage{euscript}

\topmargin  0.0in
\headheight  0.15in
\headsep  0.15in
\footskip  0.2in
\textheight 8.45in

\oddsidemargin 0.56in
\evensidemargin \oddsidemargin
\textwidth 5.8in

\pagestyle{myheadings}
\markright{\bf Sponge Layer Module User Guide}

\begin{document}


\title{Sponge Layer Module User Guide.}

\author{ Michael Nucci }

\maketitle
\section{Sponge Layer}

The governing flow equations in Loci/CHEM are shown below. The source term $\Dot{W}$ is zero for nonreacting flow, and contains the species sources for reacting flow.

\begin{equation}
\frac{d}{dt} \int_{\Omega_c(t)} Q ~dV +
\int_{{\partial\Omega_c}(t)} (F_i-F_v) ~dS =
\int_{\Omega_c(t)} \Dot{W} ~dV,
\label{gov_integral}
\end{equation}

The sponge layer model \cite{Mani} augments the source term $\Dot{W}$ as shown below. These source terms attempt to drive the flow to a user defined reference state. The damping coefficient $\sigma$ is an inverse time scale and determines the strength of the sponge.

\begin{equation}
\dot{W} =
\begin{bmatrix}
\sigma \left( \rho_{1_{ref}} - \rho_1 \right) \\
\vdots \\
\sigma \left( \rho_{s_{ref}} - \rho_s \right)\\
\vdots \\
\sigma \left( \rho_{{\mathbf{NS}}_{ref}} - \rho_{\mathbf{NS}} \right) \\
\sigma \left(\rho_{ref} \Tilde{u}_{ref} - \rho \Tilde{u} \right) \\
\sigma \left(\rho_{ref} e_{0_{ref}} - \rho e_0 \right)
\end{bmatrix}
\end{equation}

The value of $\sigma$ varies throughout the domain as a function of the distance to the nearest sponge boundary $D_s$, the length of the sponge layer $l_s$, and the maximum value of $\sigma_{max}$.

\begin{equation}
\sigma =
\begin{cases}
  \sigma_{max} \left( 1 - \frac{D_s}{l_s}\right),& \text{if } D_s < l_s \\
  0,& \text{otherwise}
\end{cases}
\end{equation}

\section{Sponge Layer Module Usage}

To access the sponge layer model, load the {\tt sponge} module.

\begin{verbatim}
loadModule: sponge
\end{verbatim}

Boundary conditions that allow outflow ({\tt outflow, outflowNRBC, extrapolate, farfield, supersonicOutflow, fixedMassOutflow}) may be augmented with the {\tt sponge} tag.

\subsection{Sponge Layer Inputs}

The {\tt sponge} module has several inputs that may be specified inside of the {\tt sponge} options list.

  \begin{list}{}{}

  \item {\tt p, T, rho}

    Two of pressure, temperature, and density must set. These scalar variables help to form the reference flow state that the sponge layer is trying to reach.
    
  \item {\tt u, M}
  
    One of either velocity or Mach number must be set. This vector variable helps to form the reference flow state that the sponge layer is trying to reach.
  
  \item {\tt length}
     
    This required input specifies the region of influence for the sponge layer, $l_s$. The sponge layer will be active in regions that are closer to a boundary condition with the {\tt sponge} tag than this value.
    
  \item {\tt sigma}
  
	This optional variable is used to specify the maximum value of {\tt $\sigma$}, the parameter determining the sponge strength. This input is optional and the default value is {\tt 1.0}.

  \end{list}

An example input for a simulation using the sponge layer is shown below.

\begin{verbatim}
	// tag boundary that permits outflow as a sponge
	boundary_conditions: <BC_1=outflow(p=101300 Pa, sponge), ...>
	      
	// sponge layer setup
	sponge: <p=101300 Pa, T=300 K, u=[100 m/s, 0,0], length=0.1 m, sigma=50000>
\end{verbatim}


\subsection{Sponge Layer Outputs}

Loading the {\tt sponge} module provides the option for outputting some variables associated with the sponge layer calculations. To output any of the following variables, add them to the \verb!plot_output! variable in the {\tt .vars} file.

  \begin{list}{}{}
  	
  	
  	\item {\tt spongeDistance}
  	
  	The distance to the nearest boundary marked with the {\tt sponge} tag.
  	
  	\item {\tt spongeSigma}
  	
  	The local value of the {\tt $\sigma$} parameter controlling the sponge layer strength.
  	
  \end{list}


\clearpage



\bibliographystyle{abbrv}
\bibliography{references}


\end{document}




